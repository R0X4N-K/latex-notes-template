\chapter{Test dei blocchi}

\section{Concetti base}

\begin{definizione}{Alfabeto}{alfabeto}
Un \term{alfabeto} è un insieme finito e non vuoto di simboli, tipicamente denotato con $\Sigma$.
\end{definizione}

\begin{teorema}{Chomsky}{chomsky}
Ogni grammatica regolare genera un linguaggio regolare.
\end{teorema}

\begin{dimostrazione}[del Teorema~\ref{thm:chomsky}]
Schizzo: costruisci un automa a stati finiti equivalente alla grammatica,
con stati che rappresentano i non terminali.
\end{dimostrazione}

\begin{lemma}{Lemma d'esempio}{finito-regolare}
Se un linguaggio è finito, allora è regolare.
\end{lemma}

\begin{dimostrazione}[del Lemma~\ref{lem:finito-regolare}]
Costruisci un AFD che riconosce l’unione finita di stringhe: ogni stringa accettata
corrisponde a un cammino verso uno stato finale distinto.
\end{dimostrazione}

\begin{proposizione}{Chiusura rispetto all’unione}{chiusura-unione}
L’insieme dei linguaggi regolari è chiuso rispetto all’unione.
\end{proposizione}

\begin{corollario}{Dall’intersezione al complemento}{intersezione}
Poiché i linguaggi regolari sono chiusi rispetto al complemento e all’unione,
lo sono anche rispetto all’intersezione.
\end{corollario}

\begin{assioma}{Assioma di scelta (esempio fittizio)}{scelta}
Dato un insieme di insiemi non vuoti, è possibile scegliere un elemento da ciascuno.
\end{assioma}

\section{Blocchi pratici}

\begin{esempio}{Parità di $1$}{parita1}
Per $\Sigma=\{0,1\}$, il linguaggio delle stringhe con numero pari di $1$ è regolare.
\end{esempio}

\begin{esercizio}{AFD su multipli di 3}{afd-mod3}
Progetta un AFD che riconosce le stringhe di $\{a,b\}^*$ con un numero di $a$ multiplo di $3$.
\end{esercizio}

\begin{nota}{Nota importante}
Ricorda che un automa può essere non deterministico, ma si può sempre costruire un DFA equivalente.
\end{nota}

\begin{attenzione}{Errore comune}
Non confondere \term{linguaggi regolari} con \term{grammatiche lineari destre}:
sono concetti diversi, sebbene collegati.
\end{attenzione}

\begin{suggerimento}{Strategia di risoluzione}
Quando puoi, disegna prima l’automa e poi formalizzalo con la quintupla.
\end{suggerimento}

\clearpage
