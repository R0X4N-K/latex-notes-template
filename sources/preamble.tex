% Preambolo comune
\usepackage{fontspec}      % LuaLaTeX
\usepackage{graphicx}
\usepackage{xcolor}        % Colori
\usepackage{polyglossia}   % Lingue con LuaLaTeX
\setmainlanguage{italian}  % <-- lingua italiana (Capitolo, Indice, data, ecc.)

% --- Font con fallback automatici ---
% Serif
\IfFontExistsTF{Libertinus Serif}{
  \setmainfont{Libertinus Serif}
}{
  \setmainfont{Latin Modern Roman} % fallback TeX Live
}
% Sans
\IfFontExistsTF{Libertinus Sans}{
  \setsansfont{Libertinus Sans}
}{
  \setsansfont{Latin Modern Sans} % fallback
}
% Mono
\IfFontExistsTF{JetBrains Mono}{
  \setmonofont{JetBrains Mono}[Scale=0.92]
}{
  \IfFontExistsTF{Fira Code}{
    \setmonofont{Fira Code}[Scale=0.92, Contextuals={Alternate}]
  }{
    \setmonofont{Latin Modern Mono}[Scale=0.92]
  }
}

\usepackage{hyperref}
\usepackage{amsmath,amssymb,amsthm}
\usepackage{titlesec}
\usepackage{tocloft}
\usepackage{fancyhdr}
\usepackage{tcolorbox}
\tcbuselibrary{theorems,skins,breakable} % per i blocchi custom

% Margini e impaginazione
\usepackage[a4paper,margin=2.5cm]{geometry}

% --- Matematica extra ---
\usepackage{mathtools}

% --- TikZ per automi/grafi ---
\usepackage{tikz}
\usetikzlibrary{automata,positioning,arrows.meta}
\tikzset{
  >=Stealth,
  node distance=2cm,
  every state/.style = {draw, thick, fill=uninaPrimary!5},
}

% --- Listings (evidenziazione codice) ---
\usepackage{listings}
\lstdefinestyle{uninastyle}{
  basicstyle=\ttfamily\small,
  numbers=left,
  numberstyle=\scriptsize,
  stepnumber=1,
  numbersep=8pt,
  keywordstyle=\bfseries\color{uninaPrimary},
  commentstyle=\itshape\color{black!55},
  stringstyle=\color{uninaSecondary!80!black},
  showstringspaces=false,
  breaklines=true,
  frame=single,
  framerule=0.4pt,
  rulecolor=\color{uninaPrimary!50},
  tabsize=2,
}
\lstset{style=uninastyle}


% Palette UNINA
\definecolor{uninaPrimary}{HTML}{1B415D}
\definecolor{uninaSecondary}{HTML}{D93C24}

\hypersetup{
    colorlinks=true,
    linkcolor=uninaPrimary,
    urlcolor=uninaPrimary,
    citecolor=uninaPrimary,
    pdftitle={Template Appunti},
    pdfauthor={Autore}
}

% Numerazione capitoli/sezioni e profondità TOC
\setcounter{secnumdepth}{3}
\setcounter{tocdepth}{3}

% Titoli uniformi
\titleformat{\chapter}[display]
  {\bfseries\Huge\color{uninaPrimary}}
  {\Large \chaptername~\thechapter}
  {0.6ex}
  {}
\titlespacing*{\chapter}{0pt}{0pt}{1.2em}

% Sezioni e sottosezioni: pulite, formali
\titleformat{\section}
  {\bfseries\Large}
  {\thesection}{0.6em}{}
\titleformat{\subsection}
  {\bfseries\large}
  {\thesubsection}{0.6em}{}
\titleformat{\subsubsection}
  {\bfseries\normalsize\itshape}
  {\thesubsubsection}{0.6em}{}

\renewcommand{\cfttoctitlefont}{\huge\bfseries\color{uninaPrimary}}
\renewcommand{\cftaftertoctitle}{\par\vspace{0.5\baselineskip}}


\setlength{\headheight}{14pt}
\pagestyle{fancy}
\fancyhf{} % pulisci tutto


\renewcommand{\headrulewidth}{0pt}

% Memorizza il titolo corrente del capitolo (senza maiuscole forzate)
\newcommand{\currentchaptertitle}{}
\makeatletter
\renewcommand{\chaptermark}[1]{%
  \gdef\currentchaptertitle{#1}%
}
\makeatother

% Footer: sinistra "Capitolo N — Titolo", destra numero di pagina (blu)
\newcommand{\chapfoot}{%
  \ifnum\value{chapter}>0
    \textcolor{uninaPrimary}{\chaptername~\thechapter\ — \currentchaptertitle}%
  \fi
}
\fancyfoot[L]{\chapfoot}
\fancyfoot[R]{\textcolor{uninaPrimary}{\thepage}}

% Anche le pagine "plain" (inizio capitolo, ecc.) usano lo stesso piè
\fancypagestyle{plain}{
  \fancyhf{}
  \renewcommand{\headrulewidth}{0pt}
  \fancyfoot[L]{\chapfoot}
  \fancyfoot[R]{\textcolor{uninaPrimary}{\thepage}}
}
