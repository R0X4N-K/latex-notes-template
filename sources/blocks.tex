% --- Tavolozza colori ---
\colorlet{thmFrame}{uninaPrimary}
\colorlet{thmBack}{uninaPrimary!5}

\colorlet{defFrame}{uninaSecondary!90!black}
\colorlet{defBack}{uninaSecondary!8}

\colorlet{noteFrame}{black!40}
\colorlet{noteBack}{black!4}

\colorlet{warnFrame}{orange!85!black}
\colorlet{warnBack}{orange!10}

\colorlet{suggFrame}{uninaPrimary!70!black}
\colorlet{suggBack}{uninaPrimary!12}

% --- Stile generale ---
\tcbset{
  before skip=8pt, after skip=8pt,
  boxrule=0.6pt,
  sharp corners,
  breakable,
}

% --- Numerazione centralizzata dei blocchi ---
% Cambia "section" -> "chapter" se preferisci numerare per capitolo.
\providecommand{\tcbwithin}{section}


% -----------------------
% Ambienti teorematici
% -----------------------
% Titolo OBBLIGATORIO { ... } per tutti.
% 'number within=section' e 'list inside=<nome>' stanno nelle INIT OPTIONS [ ... ].

\NewTcbTheorem[
  number within=\tcbwithin,
  list inside=definizione
]{definizione}{Definizione}{%
  enhanced,
  colback=defBack, colframe=defFrame,
  colbacktitle=defFrame, coltitle=white, fonttitle=\bfseries,
  boxed title style={sharp corners},
}{def}

\NewTcbTheorem[
  number within=\tcbwithin,
  list inside=teorema
]{teorema}{Teorema}{%
  enhanced,
  colback=thmBack, colframe=thmFrame,
  colbacktitle=thmFrame, coltitle=white, fonttitle=\bfseries,
  boxed title style={sharp corners},
}{thm}

\NewTcbTheorem[
  number within=\tcbwithin,
  list inside=lemma
]{lemma}{Lemma}{%
  enhanced,
  colback=thmBack, colframe=thmFrame!85!black,
  colbacktitle=thmFrame!85!black, coltitle=white, fonttitle=\bfseries,
  boxed title style={sharp corners},
}{lem}

\NewTcbTheorem[
  number within=\tcbwithin,
  list inside=proposizione
]{proposizione}{Proposizione}{%
  enhanced,
  colback=thmBack, colframe=thmFrame!70!black,
  colbacktitle=thmFrame!70!black, coltitle=white, fonttitle=\bfseries,
  boxed title style={sharp corners},
}{prop}

\NewTcbTheorem[
  number within=\tcbwithin,
  list inside=corollario
]{corollario}{Corollario}{%
  enhanced,
  colback=thmBack, colframe=thmFrame!60!black,
  colbacktitle=thmFrame!60!black, coltitle=white, fonttitle=\bfseries,
  boxed title style={sharp corners},
}{cor}

\NewTcbTheorem[
  number within=\tcbwithin,
  list inside=assioma
]{assioma}{Assioma}{%
  enhanced,
  colback=thmBack, colframe=thmFrame!50!black,
  colbacktitle=thmFrame!50!black, coltitle=white, fonttitle=\bfseries,
  boxed title style={sharp corners},
}{axs}

% -----------------------
% Blocchi pratici (numerati)
% -----------------------
\NewTcbTheorem[
  number within=\tcbwithin,
  list inside=esempio
]{esempio}{Esempio}{%
  enhanced,
  colback=noteBack, colframe=noteFrame,
  colbacktitle=noteFrame, coltitle=white, fonttitle=\bfseries,
  boxed title style={sharp corners},
}{ex}

\NewTcbTheorem[
  number within=\tcbwithin,
  list inside=esercizio
]{esercizio}{Esercizio}{%
  enhanced,
  colback=suggBack, colframe=suggFrame,
  colbacktitle=suggFrame, coltitle=white, fonttitle=\bfseries,
  boxed title style={sharp corners},
}{exr}


% -----------------------
% Blocchi informativi (non numerati)
% -----------------------
\newtcolorbox{nota}[1]{%
  enhanced, title={#1}, fonttitle=\bfseries,
  colback=noteBack, colframe=noteFrame,
  colbacktitle=noteFrame, coltitle=white,
  boxed title style={sharp corners},
}

\newtcolorbox{attenzione}[1]{%
  enhanced, title={#1}, fonttitle=\bfseries,
  colback=warnBack, colframe=warnFrame,
  colbacktitle=warnFrame, coltitle=white,
  boxed title style={sharp corners},
}

\newtcolorbox{suggerimento}[1]{%
  enhanced, title={#1}, fonttitle=\bfseries,
  colback=suggBack, colframe=suggFrame,
  colbacktitle=suggFrame, coltitle=white,
  boxed title style={sharp corners},
}

% -----------------------
% Dimostrazione (box) — titolo opzionale [ ... ]
% -----------------------
% Uso:
% \begin{dimostrazione}[del Teorema~\ref{thm:...}]
%   ...
% \end{dimostrazione}
\newtcolorbox{dimostrazione}[1][]{%
  enhanced, breakable,
  colback=white, colframe=thmFrame!70!black,
  colbacktitle=thmFrame!70!black,
  coltitle=white, fonttitle=\bfseries,
  % Link bianchi nel titolo (anche dentro \ref nel tuo [ ... ])
  title={\begingroup
           \hypersetup{linkcolor=white,citecolor=white,urlcolor=white}%
           Dimostrazione~#1%
         \endgroup},
  boxed title style={sharp corners},
}



% -----------------------
% Altri comandi
% -----------------------
\newcommand{\term}[1]{\textbf{\textcolor{uninaSecondary}{#1}}}

% Nessun quadratino nero automatico
\renewcommand{\qedsymbol}{}

% -----------------------
% Wrapper per stampare le liste globali (con voce in ToC)
% -----------------------
\newcommand{\listofteorema}{%
  \tcblistof[\chapter*]{teorema}{Lista dei Teoremi}%
  \addcontentsline{toc}{chapter}{Lista dei Teoremi}}

\newcommand{\listofdefinizione}{%
  \tcblistof[\chapter*]{definizione}{Lista delle Definizioni}%
  \addcontentsline{toc}{chapter}{Lista delle Definizioni}}

\newcommand{\listoflemma}{%
  \tcblistof[\chapter*]{lemma}{Lista dei Lemmi}%
  \addcontentsline{toc}{chapter}{Lista dei Lemmi}}

\newcommand{\listofproposizione}{%
  \tcblistof[\chapter*]{proposizione}{Lista delle Proposizioni}%
  \addcontentsline{toc}{chapter}{Lista delle Proposizioni}}

\newcommand{\listofcorollario}{%
  \tcblistof[\chapter*]{corollario}{Lista dei Corollari}%
  \addcontentsline{toc}{chapter}{Lista dei Corollari}}

\newcommand{\listofassioma}{%
  \tcblistof[\chapter*]{assioma}{Lista degli Assiomi}%
  \addcontentsline{toc}{chapter}{Lista degli Assiomi}}

\newcommand{\listofesempio}{%
  \tcblistof[\chapter*]{esempio}{Lista degli Esempi}%
  \addcontentsline{toc}{chapter}{Lista degli Esempi}}

\newcommand{\listofesercizio}{%
  \tcblistof[\chapter*]{esercizio}{Lista degli Esercizi}%
  \addcontentsline{toc}{chapter}{Lista degli Esercizi}}
